
%%%%%%%%%%%%%%%%%%%%%%%%%%%%%%%% SCRIPT FOR E-RARA AND MDZ FILES     %%%%%%%%%%%%%%%%%%%%%%%%%%%%%%%%%%%%%%%%%%%%%%%%
%%%%%%%%%%%%%%%%%%%%%%%%%%% fini le 30.04.2025 par F. GOY            %%%%%%%%%%%%%%%%%%%%%%%%%%%%%%%%%%%%%%%%%%%%%%%%
% !TeX TS-program = lualatex
\documentclass{article}
\usepackage[T1]{fontenc}
\usepackage{microtype}
\usepackage[pdfusetitle,hidelinks]{hyperref}

\usepackage{polyglossia}
\setmainlanguage{english}
\setotherlanguages{latin,greek}
\usepackage[series={},nocritical,noend,noeledsec,nofamiliar,noledgroup]{reledmac}
\usepackage{reledpar}

\usepackage{fontspec}
\setmainfont{TeX Gyre Termes}

\usepackage{sectsty}
\usepackage{xcolor}

\usepackage{fancyhdr}
\pagestyle{fancy}
\fancyhf{}
\fancyhead[LE,RO]{\nouppercase{\leftmark}}  
\cfoot{\thepage}
\renewcommand{\headrulewidth}{0.4pt}

% Redefine \section to remove numbering
\usepackage{titlesec}
\titleformat{\section}[block]{\normalfont\scshape\color{gray}}{}{0pt}{} 
\titleformat{\subsection}[hang]{\normalfont}{}{0pt}{} 
\titleformat{\subsubsection}[hang]{\normalfont\footnotesize\color{black}}{}{0pt}{}

\makeatletter
\renewcommand{\sectionmark}[1]{\markboth{#1}{}} 
\renewcommand{\subsectionmark}[1]{\markright{#1}} 
\renewcommand{\numberline}[1]{} 
\makeatother

\begin{document}

\date{}
\title{Commentarii In Epistolas D. Pavli Ad Timotheum : [Bullinger, Heinrich], [1536]}
\maketitle
\tableofcontents
\clearpage
\begin{pages} 
\beginnumbering

\endnumbering\beginnumbering\section{FRCAOM06\_COLE\_241062A\_0804}\pstart
\\
Brussel (Pierre-Louis),\\
dit la Pointe,\\
natif du Havre,\\
Soldat au régiment du Cap-français,\\
à Saint-Domingue\\
Sa femme Anne Saty demande à le\\
rejoindre\\
1776
\pend
\endnumbering\beginnumbering\section{FRCAOM06\_COLE\_241062A\_0805}\pstart
\\
Nous soussignés Capitaine d'infanterie aide major\\
chargé du Détail du Régiment du Cap certifions que\\
Le nommé Louis Brusset dit La Pointe fils de\\
Barthelemy et de Marie Jeanne La Lande natif du\\
havre agé de 28 ans taille de 5p. 3p. cheveux et\\
sourcils Bruns les yeux Roux le visage rong, Grenadier\\
de la compagnie du Ch\^{}er Dallard a contracté un\\
engagement pour huit ans au Cap le 21 fevrier\\
1774 en foy dequoi nous lui avons délivré le\\
present pour lui servir et valoir en ce que de raison\\
fait au Cap le 28 may 1775\\
Dufour
\pend
\endnumbering\beginnumbering\section{FRCAOM06\_COLE\_241062A\_0806}
\vspace{0.5cm}\noindent
\textit{mg}
\footnotesize \\
1776\\
M. D. L. C※\\
n°49
\normalsize \pstart
\\
Monseigneur de Sartine\\
Ministre et Secrétaire d'État\\
\& la marine\\
M. Reilleur 20 août\\
Ecrit a m. Mistral au havre le 23 aout\\
1776\\
Suplie très humblement Anne\\
Victoire Saty, demeurante au havre de grace\\
et prend la tres respectueuse liberté de\\
représenter a Votre Grandeur\\
Monseigneur, que Pierre Louis\\
Brussel, Son mary, ayant voulu essayer\\
de la navigation, mais sa constitution ne luy\\
ayant pas permis de pouvoir suivre cet\\
état, il auroit, ainsi qu'il le conste
\pend\pstart
\\
La f\^{}e Brussel\\
Pierre Louis\\
A Monseigneur
\pend
\endnumbering\beginnumbering\section{FRCAOM06\_COLE\_241062A\_0807}\pstart
\\
par le certificat cy joint de M. Dufour,\\
capitaine d'Infanterie, ayde major chargé\\
du détail du Régiment du Cap, pris\\
le party de servir Sa Majesté dans ses\\
troupes des colonies. Ce soldat désirant\\
avoir près de luy la supliante son épouse\\
et une petite fille quelle a agé de 6 Ans\\
de leur mariage; mais elle ne peut, par rapport\\
a l'Indigeance dans laquelle elle est,\\
satisfaire au désir qu'à son mary de\\
la posseder; C'est pourquoy elle a recours\\
a vos bontés, Monseigneur, pour que\\
vous daignés luy accorder son passage\\
pour le Cap ainsi que celuy de sa petite\\
fille.
\pend
\endnumbering\beginnumbering\section{FRCAOM06\_COLE\_241065A\_0010}\pstart
\\
Cabaret (le nommé)\\
Soldat au regiment de l'Amérique\\
on demande pour lui son congé\\
1775
\pend
\endnumbering\beginnumbering\section{FRCAOM06\_COLE\_241065A\_0011}
\vspace{0.5cm}\noindent
\textit{mg}
\footnotesize \\
Colonies 1775.\\
Cabaret\\
Congé absolu.\\
Pour le N\^{}e Cabaret.\\
R le 12 avril 1775
\normalsize \pstart
\\
Mémoire\\
Monseigneur\\
M. aullain\\
Les Parens du N\^{}e Cabaret\\
soldat au Reg\^{}t de l'amérique\\
viennent de deposer 200\^{}lt entre les\\
mains de M\^{}e Agobert pour son\\
congé absolu, qu'ils m'ont prié de\\
demander étant absoluement necessaire\\
a son pere, et je vous supplie de le\\
lui accorder ※\\
Genlich
\pend
\vspace{0.5cm}\noindent
\textit{mg}
\footnotesize \\
à classer
\normalsize 
\endnumbering\beginnumbering\section{FRCAOM06\_COLE\_241065A\_0012}
\vspace{0.5cm}\noindent
\textit{mg}
\footnotesize \\
B\^{}au des fonds\\
Cabarres
\normalsize \pstart
\\
Par Décision du 11 Juillet 1783. Monsiengeur\\
a accordé une gratification de 600\^{}lt au S.\\
Cabarres député de la nation francoise au\\
Port du Passage, en considération des soins qu'il\\
s'est donné pour le relevement de la Gabarre\\
le Portefaix echouée en cette rade et de l'économie\\
qu'il a apporté dans cette opération.\\
La Fresnayes
\pend\pstart
\\
v. E. 58
\pend
\endnumbering\beginnumbering\section{FRCAOM06\_COLE\_241202A\_0498}\pstart
\\
Florac (Pierre)\\
Soldat au régiment de Karrer, a Rochefort\\
1747
\pend
\endnumbering\beginnumbering\section{FRCAOM06\_COLE\_241202A\_0499}
\vspace{0.5cm}\noindent
\textit{mg}
\footnotesize \\
Florac
\normalsize \pstart
\\
M. Karrer 26. aoust 1747\\
Monseigneur\\
M de la Sons\\
a labonne heure.
\pend\pstart
\\
Il est vray que le nommé pierre florac est soldat de ma Colonelle,\\
qui m'ayant demandé en juillet de l'année dernière un Congé de semestre\\
pour vaquer a des affaires de famille, je luy en fis expedier un a\\
l'expiration duquel il n'a pas rejoint; aquoy je n'ay fais nulle\\
attention, attendu, Monseigneur, qu'ayant Cÿ devant entreveu\\
quelque dérangement dans l'esprit de ce soldat agé, je l'ay facilement\\
perdu de vue; et comme sa déclaration Volontaire ne peut être regardée
\pend
\endnumbering\beginnumbering\section{FRCAOM06\_COLE\_241202A\_0500}\pstart
\\
qune Suite de le même dérangement qui le mettroit hors du Cas d'être jugé\\
Comme déserteur, je ne prévois pas, Monseigneur, qu'il soit nécéssaire\\
d'occasionner une dépense au Roy pour le faire Conduire au Régiment, me\\
Soumettant au surplus a toucequil plaise a Votre Grandeur ordonner\\
a ce sujet.\\
La Santé duS\^{}r Marquis enseigne de la quatrieme Compagnie étant parfaitement\\
retablie, je me propose, Monsiegneur, de le faire passer a sa troupe par le\\
premier vaisseau qui partira pour la louisianne, j'ose dont supplier tres\\
humblement Votre Grandeur de vouloir bien donner les ordres nécéssaires pour\\
son passage et Celuy de vingt Soldats que je destine pour Remplacer ceux de Cette\\
Compagnie a qui le Congez est dû.\\
Je suis avec un profond Respect\\
Monseigneur\\
De Votre Grandeur\\
Le tres humble et très\\
à Rochefort le 26 aoust 1747\\
obeissant serviteur\\
DeKarrer
\pend
\endnumbering\beginnumbering\section{FRCAOM06\_COLE\_241235A\_0558}\pstart
\\
Gorry\\
(Jean)\\
Soldat en Guyane\\
+ 1788\\
Colonies\\
E 209
\pend
\endnumbering\beginnumbering\section{FRCAOM06\_COLE\_241235A\_0560}
\vspace{0.5cm}\noindent
\textit{mg}
\footnotesize \\
Gorry\\
(Jean)\\
2\^{}ta\\
n°3.
\normalsize \pstart
\\
Extrait du Registre\\
des Successions\\
Aujourd'huy Vingt neuf octobre mil sept\\
Cent quatre huit, ayant été informé que le nommé\\
Jean Gorry fils de Jacques Gorry et de Thereze\\
Gorlane né à Branguetol en Languedoc L'an 1758\\
fusilier de la Compagnie descoublant n°4 au\\
Bataillon de la guianne etoit mort au porte d'oyapock\\
ou il etoit detaché Le 25. du present mois\\
Nous Quartier maitre Tresorier dudit Bataillon,\\
nous sommes Transportés aux Cazernes, ou son Coffre\\
avoit été envoyé, et La, en presence du M M. Dutraque\\
Lieutenant Commandant Ladite Compagnie \& De Titty;\\
Ecrivain ordonaire des Colonies pour La maladie du M\\
Boüé ecrivain principal chargé de la police du Bataillon\\
nous avons procédé à L'inventaire des effets dudit Gorry,\\
Lesquels ont été Vendus sur le Champs au plus offrant\\
et dernier encherisseur Comme suit.\\
Savoir.\\
Un Garreau ... "\^{}lt, 12\^{}s, "\^{}d.\\
Une grande Culotte... ",10,".\\
Deux Gillets \& Calotes de Coutil...", 4, 4.\\
Deux idem en Toile... 3,16,".\\
9\^{}lt, 2\^{}s,"\^{}d.
\pend
\endnumbering\beginnumbering\section{FRCAOM06\_COLE\_241235A\_0561}\pstart
\\
De L'autre Part,...9\^{}lt,2\^{}d,"\^{}s\\
dce\\
Une Culote idem...",8,".\\
Quatre paires guetres blanches... 2,6,".\\
Une chemise...",8,".\\
Trois paires bas de fil...2,8,".\\
Sept Cols de Basin...1,6,".\\
Une Broche \& un Bonet...",6,".\\
Une paire souliers et une paire boucles...2,14,".\\
Un Coffre...6,",".\\
En Especes... 98,10,".\\
Formant une Somme de Cent\\
123\^{}lt,8\^{}d,"\^{}s.\\
vingt trois Livres huit Sols.\\
De laquelle Somme Celle provenant des\\
Effets vendus sera passée en Compte aux hommes\\
qui les ont achetés, celle trouvée en Especes à été\\
déposée à la Caisse pour être envoyée au Ministre\\
à la prochaine Revue d'inspection\\
à Cayenne les jours, mois \& an que dessus,\\
signé sur le Registre Dutrague, Detitty,\\
Et Geslin quartien m\^{}tre T\^{}r※\\
Pour Copie Conforme\\
a l'original.\\
Boüé,
\pend
\endnumbering\beginnumbering\section{FRCAOM06\_COLE\_241235A\_0562}
\vspace{0.5cm}\noindent
\textit{mg}
\footnotesize \\
1\^{}er\\
N°3
\normalsize \pstart
\\
Extrait du Registre\\
des Successions\\
Aujourdhuy vingt neuf octobre mil Sept cent\\
quatre vingt huit ayant été informé que le nomé jean\\
Gorry fils de jacques Gorry et de Therese Gortand né\\
à Brangnetot en Languedoc l'an 1758, fusilier de la\\
Compagnie D'Escoublant n° 4. au bataillon de la\\
Guigné Etoit mort au poste D'oyapock ou il etoit\\
Detaché le 25. du present mois\\
nous sommes transportés aux cazernes ou son coffre\\
Nous quartier maitre tresorier du dit bataillon\\
avoit ⟦été⟧ envoiyé \& l'a en presence de MM Dutraque\\
Lieutenant Commandant la dite Compagnie \& de\\
tilly ecrivain ordinaire des Colonies, pour la maladie\\
\& M\^{}e Boué Ecrivain principal chargé de la police\\
du Bataillon nous avons procedé a linventaire des\\
effets du dit Gorry Lesquels ont été vendus sur\\
Le champs au plus offrant et dernier\\
encherissemens comme suit\\
un Sarreau... " 12\\
savoir.\\
une grande culotte... " 10\\
deux Gillets \& culottes de coutil... 4, 4\\
deux idem en toille ... 3, 16\\
une culotte idem... " 8\\
quatre p\^{}e Guettres blanches ... 2...6\\
une chemise ... ", 8\\
Trois p\^{}e bas de fil... 2, 8\\
sept cols de basin...5, 6\\
une brosse \& un bonnet..... ",6\\
une  p\^{}a souliers \& une p\^{}e boucle 2, 14\\
un coffre..., 6\\
En especes.... 98,10\\
Soixant une somme de cent vingt\\
trois livres huit sols\\
23\^{}lt 8\^{}s
\pend
\endnumbering\beginnumbering\section{FRCAOM06\_COLE\_241235A\_0563}\pstart
\\
A Laquelle somme celle provenant des\\
effets vendus sera passée en compte aux hommes\\
qui les ont achetes cette trouvée en especes à\\
été déposée à la Caisse pour envoyer au ministre\\
a la prochaine Revue d'inspection\\
A Caïenne les jours mois \& an que dessus\\
Signé sur le Registre Dutraque Detilly\\
Geslius quartier maitre tresorier\\
Pour Copie conforme\\
a L'original.\\
Boüé\\
C\^{}7126
\pend
\endnumbering\beginnumbering\section{FRCAOM06\_COLE\_241342A\_0507}\pstart
\\
Hiot (François),\\
dit Duchâteau,\\
natif de Champagne,\\
soldat d'artillerie à Saint-Domingue\\
mort noyé\\
1765
\pend
\endnumbering\beginnumbering\section{FRCAOM06\_COLE\_241342A\_0508}
\vspace{0.5cm}\noindent
\textit{mg}
\footnotesize \\
S\^{}t Domingue
\normalsize 
\vspace{0.5cm}\noindent
\textit{mg}
\footnotesize \\
10.février\\
1765.\\
N°128
\normalsize \pstart
\\
An mil Sept cent soixante cinq le dixieme\\
jour de février, sur les six heures du soir, par devant\\
nous Commissaire des guerres, Employé au\\
Département du Cap Isle S\^{}t Domingue, est\\
comparû, accompagné de son Capitaine, Nom̃é\\
nicolas Emond dit Emond, natif de Rencourt\\
juridiction de Bar-Le-Duc, Sergent de la c\^{}ie de la Ronde,\\
Brigade de Beausire du Corps-Royal, détaché à\\
S\^{}t Domingue, où il sert en qualité de premier\\
sergent dans la Compagnie de Dulac Postiche dud.\\
Corps. Royal, lequel a dit et déposé, que le même jour\\
Il serait Sortie, sur les une heure après midy, du\\
quartier pour s'aller promener en compagnie de\\
françois hiot d\^{}t Duchateau, natif de Voussy,\\
Juridiction de Rhetel-Mazarin en Champagne\\
Sergent de la Compagnie de Detiziers de la même\\
Brigade, de Beausire, aussi détaché à S\^{}t Domingue,\\
et servant dans la même Compagnie que Le Déposant;\\
qu'étant passé auprès de l'Arcenas, Ils auraient suivi\\
le Bord de la Mer, jusqu'au fort de Licolet, qu'ils\\
auraient tourné sans y entrer, pour continuer de\\
reconnaitre le bord de la Mer du côté du Nord, et\\
voir par par mouvement de curiosité, les bateries\\
qui y sont construittes; que suivant le Bord de la Mer\\
ils seraient arrivés a porté de fusil de la Baterie de la
\pend\pstart
\\
N°6\\
hiot dit Duchâteau\\
Pour le M\^{}e Duchâteau
\pend
\endnumbering\beginnumbering\section{FRCAOM06\_COLE\_241342A\_0509}\pstart
\\
droite de la Bande du Nord, que pour lors luy déposant\\
se serait arrêté, pour attendre son Camarade, qui était\\
environ à trente pas derriere luy; qu'étant en cette\\
situation appuyé sur un baton qui luy servait de\\
canne, le dos tourné à la Mer, il se serait senti tout\\
à coup enlevé par une Lamme, qui l'aurait Entrainé:\\
qu'avant quil pût se reconnaitre et nager, pour\\
regagner le Rivage, il s'etait vû à trente pas au moins\\
de l'endroit d'où il avait été enlevé; qu'ayant fait\\
plusieurs Efforts pour revenir à terre, il avait eû\\
beaucoup de peine à y parvenir, la Mer le jettant\\
toujours contre un Rocher, où il s'est fait plusieurs\\
blessures, et meurtrissures; et qui a entierement déchiré\\
sa Chemise; Culotte, bas, et le reste de ses hardes.\\
qu'ayant enfin été assés heureux, pour ratraper\\
Le Rivage, son premier soin en revenant à luy avait\\
été de regarder s'il verrait son Camarade, qu'il l'avait\\
appelé à plusieurs reprises, mais que ne l'apercevant\\
pas, il n'aurait pas douté un instant, comm'il ne\\
doute pas encore, qu'il n'eût été enlevé par la même Lamme,\\
oû que voulant le secourir, il se soit noyé. qu'enfin\\
voyant que ses recherches étaient inutiles, il se serait\\
trainé toujours le long de la mer, en passant la\\
premiere et seconde Baterie, jusqu'à l'habitation de\\
M. Raby, qui n'est pas éloignée de cette dernière; que\\
y étant entré pour demander quelques Secours, on\\
Luy avait donné entierement dequoy se changer, toutes
\pend
\endnumbering\beginnumbering\section{FRCAOM06\_COLE\_241342A\_0510}\pstart
\\
ses hardes étant en Lambeaux; et qu'ayant\\
raconté son avanture à l'Oeconome qu'il avait\\
trouvé sur l'habitation, il l'avait prié d'envoyer\\
voir, s'il ne pourait point avoir des nouvelles de son\\
Camarade: qu'enfin ayant repris un peu ses forces,\\
il avait taché de regagner la ville où il était\\
arrivé.\\
Lecture à luy faite de sa présente déposition,\\
a dit qu'elle contient Vérité, et qu'il n'a rien à\\
y ajouter; n'y a retrancher: et a signé avec nous\\
et M. Dulac son Capitaine, en présence de qui elle\\
été faite.\\
Au Cap-français Isles et Côte S\^{}t Domingue, les\\
jour, mois et an que dessus ※\\
Le Ch Duzat de monterau\\
Millin De Granmaison\\
emous\\
Et ce jourd'huy, quinziéme jour de mars de la\\
même année mil sept cent Soixante cinq, attendu\\
que depuis le Dix février on n'a plus eû de nouvelles\\
du nommé Duchateau, et qu'on ne peut plus douter,\\
qu'il soit réellement noyé; nous Commissaire des\\
Guerres avons dressé le présent Procés-verbal, pour\\
servir à constater son décés; dont nous avons remis
\pend
\endnumbering\beginnumbering\section{FRCAOM06\_COLE\_241342A\_0511}\pstart
\\
une Expédition à M. Dulac, son Capitaine, pour\\
être Envoyée à sa Brigade, et en avons adressé une\\
autre au Bureau de la guerre, pour y être délivrée\\
à la famille du dit déffunct, et tenir lieu d'extrait\\
Mortuaire.\\
Fait \& Arrêté par nous Commissaire des guerres\\
Susdit au Cap les jours, mois et an que dessus ※\\
Millin de Grandmaison
\pend
\endnumbering\beginnumbering\section{FRCAOM06\_COLE\_241343A\_0226}\pstart
\\
Honoré (Charles)\\
Colonies\\
E 224-
\pend
\endnumbering\beginnumbering\section{FRCAOM06\_COLE\_241343A\_0227}
\vspace{0.5cm}\noindent
\textit{mg}
\footnotesize \\
L. du 8 9\^{}bre 1820\\
N°13.\\
honoré
\normalsize \pstart
\\
Honoré.\\
Extrait des Régistres\\
de l'Etat civil de la ville de\\
Cayenne en dépôt au Greffe du\\
Tribunal de première Instance,\\
dela Guyane française séans\\
à Cayenne.\\
L'an Mil sept cent trente sept le trente\\
septembre, Charles honoré dit la Verdure natif\\
de Paris paroisse de S\^{}t Mort fils de N. honoré\\
vitrier et de Marie andré, soldat de la Garnison\\
décédé aujourd'hui à l'hopital a été inhumé dans\\
le cimetière de cette paroisse. signé J. D. Panier\\
prêtre.
\pend
\vspace{0.5cm}\noindent
\textit{mg}
\footnotesize \\
Vû pour Légatisation de la Signature\\
de M. Frachon\\
Paris le 31. Octobre 1820.\\
Pour le Ministre Secrétaire d'Etat de la\\
marine et des Colonies par autorisation.\\
Le secrtétaire Général du Ministère\\
V. Vauvillier
\normalsize \pstart
\\
Pour expédition Conforme\\
signé Lagrance Greffier\\
Pour Copie Conforme\\
Le sécrétaire archiviste.\\
Frachon
\pend
\endnumbering\beginnumbering\section{FRCAOM06\_COLE\_241349C\_0106}
\vspace{0.5cm}\noindent
\textit{mg}
\footnotesize \\
N°20.
\normalsize \pstart
\\
Garde d'artillerie à la Martinique\\
Joly (Jean-Baptiste)\\
⟦1783-1786⟧\\
1760-1786\\
E. 231
\pend
\endnumbering\beginnumbering\section{FRCAOM06\_COLE\_241349C\_0107}\pstart
\\
Joly\\
Garde-magasin d'artillerie\\
à la Martinique\\
1768.\\
Colonies\\
E 231
\pend
\endnumbering\beginnumbering\section{FRCAOM06\_COLE\_241349C\_0108}
\vspace{0.5cm}\noindent
\textit{mg}
\footnotesize \\
Pour le S\^{}r Joly\\
Garde D'artillerie
\normalsize \pstart
\\
Le S\^{}r De La Rigaudie Commandant de l'artillerie a la\\
martinique a l'honneur de representer a Monsieur Le\\
Ch\^{}er du S\^{}t Maurie que M\^{}r de la Roüe ayant Jugé\\
Indispensable un garde magazin D'artillerie a S\^{}t\\
Pierre, Commis le S\^{}r Joly pour en faire les fonctions\\
avec un peu argent du pain, jusque ace que la Cour\\
aÿe approuvé cette Commission avec promesse que ses\\
appointemente augmenteroient sur la Satisfaction que\\
L'on auroit de ses Services, depuis ce temps ces appointemens\\
n'ont point augmenté quoyque pourveu de Commission\\
neamoins le zele et lactivité dudit Joly ne sest point\\
ralentie, cest un Sujet qui merite beaucoup par ses\\
services presents, et passer en qualité de Commissaires de\\
la Marine depuis 25 ans; a ces causes je supplie\\
très humblement, Monsieur, le Ch\^{}er de S\^{}t Maures de\\
Concerver un sujet aussy bon q'utile au Roy, en luy donnãs\\
les moyens de Continuer Ses Services et de Subsister par\\
une augmentation d'appointements Convenable au prises\\
et soins que sa place luy reafirme, joint a la Chereté du\\
pain, ce sera le moyen de Conserver au Roy un Sujet\\
sur lequel a moult egard, lon peut conter, et que nous\\
estimont tous※\\
au fort Royal Ce 16 May 1786.\\
Eunigaudes\\
Il est bien vray que le S\^{}r Joly sert avec exactitude inteligence\\
et probité et quil a peu de moyen pour vivre avec un écu\\
par jour je pensse qu'en Luy donnant 150\^{}lt par mois lon ne peut luy\\
donner moins mais quil doit detre satisfait je prie monsieur le\\
president de prinier de vouloir bien s'interesser a sa faveur il esgne\\
le sujet est tant dire Le M de Saint maures
\pend\pstart
\\
Mémoire.\\
Joly.
\pend
\endnumbering\beginnumbering\section{FRCAOM06\_COLE\_241349C\_0109}\pstart
\\
joint a la lettre de M. de Peynier du 30. may 1768
\pend
\endnumbering\beginnumbering\section{FRCAOM06\_COLE\_241349C\_0110}
\vspace{0.5cm}\noindent
\textit{mg}
\footnotesize \\
ordrede L\^{}t de Port au fort Royal\\
de la Martinique\\
Pour le s\^{}r LeSage
\normalsize \pstart
\\
a V\^{}les Le 21. 9\^{}bre.1767.\\
N. 14\\
De Par le Roy
\pend\pstart
\\
S\^{}t Sa Majesté Estimant necess\^{}re de commettre une personne\\
capable et exprimentée au fait de la Marine pour faire\\
les fonctions de Lieutenant de Port au fort Royal de\\
La martinique et sachant que le S\^{}r\\
Le Sage a les qualités ness\^{}re pour s'en bien acquitter,\\
Sa Majesté l'a retenu et ordonné, retient et ordonne Lieutenant\\
de Port au fort Royal pour y servir suivant cette\\
distinction sous l'autorité du Gouverneur son Lieutenant\\
general et de l'Intend\^{}t des Isles de la Martinique et\\
de St Lucie auxquel mande sa Majesté de le recevoir\\
et de le faire reconnaitre en la d\^{}e qualité de tous ceux et\\
ainsi qu'il appartiendra\\
fait a versailles le 21. novembre 1767.
\pend
\vspace{0.5cm}\noindent
\textit{mg}
\footnotesize \\
Ordre de garde magazin particulier\\
d'artillerie au fort S\^{}t Pierre de la martinique\\
Pour Le S\^{}r Joly
\normalsize \pstart
\\
De par le Roy\\
Sa Majesté⟦⟧ voulant faire choix d'une personne\\
capable et fidele pour remplir la place de garde magazin\\
particulier d'artillerie au fort S\^{}t Pierre de La martinique\\
et sachant que le S\^{}r\\
Joly\\
a les qualités necess\^{}res pour en bien executer les fonctions,\\
sa m\^{}t. l'aretant et ordonné garde magazin particulier\\
d'artillerie au fort S\^{}t Pierre pour en lad\^{}e qualité prendre
\pend
\endnumbering\beginnumbering\section{FRCAOM06\_COLE\_241349C\_0111}\pstart
\\
sous sa garde et par inventaire tous les efforts et munitions\\
qui concerne l'artillerie, tant pour les Registres\\
necess\^{}re d'Entrée et desortir et se conforme d'ailleur aux\\
ordres qu'il recevra du Gouverneur Lieutenant gn\^{}al des\\
Isles de la Martinique et de S\^{}t Lucie, pour ledit Employ\\
exder aux appointement qui lui seront reglé par les etats\\
de ordonnances de sa m\^{}t Mandé sa Majesté au Gouvernens\\
Lieutenant gn\^{}al des d. Isles de ma martinique et de S\^{}te Lucie\\
de faire reconnaitre ledit S\^{}r Joly en lad\^{}e qualité\\
de tout ceux et ainsy qu'il appartiendra.\\
fait a Versailles le 21. novembre 1767.
\pend
\endnumbering\beginnumbering\section{FRCAOM06\_COLE\_241349C\_0112}
\vspace{0.5cm}\noindent
\textit{mg}
\footnotesize \\
Artillerie\\
Joly\\
des\\
Colonies\\
Le 23 Fevrier 1786.\\
Expéditions dattées\\
du 11 mars
\normalsize \pstart
\\
recommandation de M. Du Puget, demande\\
M de Manson d'après la\\
la Médaille d'or, telle qu'on l'accorde dans\\
l'artillerie de France, Pour le M\^{}e Jean B\^{}te\\
Joly, Garde d'Artillerie à S\^{}t Pierre de\\
La Martinique.\\
n'a point céssé par son zele et sa bonne\\
Cet employé qui sert depuis 1746,\\
conduite, de se rendre digne de la récompense\\
sollicitée pour Luy.\\
Monseigneur est, En conséquence,\\
Supplié de faire connoitre\\
ses intentions※\\
approuvé
\pend\pstart
\\
Mémoire
\pend
\endnumbering\beginnumbering\section{FRCAOM06\_COLE\_241349C\_0113}\pstart
\\
Double\\
S\^{}t Pierre Martinique Le p\^{}er Avril 1783\\
Collonies. Garde Magazin de L'artillerie\\
a S\^{}t Pierre\\
⟦M. Votre⟧ M. Vilain\\
Mémoire.\\
Pour servir a la Demande de la Médaille-\\
J B\^{}te Joly natif de perpignan en Rosillon Pendent Le Susdit temps il a assisté\\
agé de 54 ans, après avoir fait les a la Bataille de plaisance, a la Retraite\\
dernieres Campagnes Ditalie dans les de torture ou il fut prisonnier de Guerre,\\
Volontaire, de Ganles a Commencer de et A la Martinique il a envoÿé les\\
L'année 1746 est passé a la Martinique deux dernieres guerre dont a la suitte\\
et a été incorporé l'an 1750 dans les de la premiere il a subi Le fort de\\
Comapgnies franches des Cannoniers L'isle comme il paroit de L'autre part\\
Bombardiers détachés de la Marine de et depuis son retour il a Continué\\
Rochefort\\
Son service jusques a ce jour ce quil\\
L'an 1756, il a été fait Sergent. Le porte a sollicité La Medaille\\
L'an 1760 il a été chargé du Detaille qui servira de titre a son fils agé\\
de la Compagnie de pelletier a S\^{}t Pierre de 15 ans pour pouvoir meritter\\
et a été pouvu d'une Commission dans La suitte des bienfaits\\
de metre Cannonicer au dit Département du Roy comme fils d'un\\
et a continué jusques a la prise encien serviteur.\\
de lisle quil est repassé en France\\
avec sa Compagnie et Licencié à Rochefort\\
L'année 1762.\\
JB Jolÿ\\
L'an 1763 en Conséquence des ordres du Nous commandant en Chef L'artillerie\\
Ministre il est repassé à la Martinique de la de la M.S. que des iles du vent, croyons que\\
Et il a été placé Garde Dartillerie a S\^{}te Le S\^{}t Joly est Suceptible, patr son zêle et\\
Pierre ou il a Continué Son Service\\
jusques a Ce jour.\\
L'ancienneté de ses services de la Médaille\\
qi'il demande.\\
grandedrer
\pend
\endnumbering\beginnumbering\section{FRCAOM06\_COLE\_241349C\_0114}\pstart
\\
Dupl\^{}a\\
S\^{}t Pierre Martinique Le 1\^{}er avril 1785\\
Collonies\\
Garde magasin\\
de L'artillerie\\
A S\^{}t Pierre\\
Pour Servir à la demande de la Médaille\\
Jean Baptiste Joly natif de
\pend\pstart
\\
Perpignan en Rossillon agé de\\
56 ans, apres avoir fait des\\
dernieres campagnes ditalie dans\\
Les volontaires de Gantés a Commencer\\
de l'année 1746 et passé à la MSq̃ue\\
est à été Incorporé l'an 1750 dans\\
Les Compagnies franches des Canoniers\\
Bombardiers detachés de la Marine\\
de Rochefort.\\
L'an 1753 il à été fait Sergent.\\
L'an 1760 il a été chargé du detail\\
de la Compagnie de Belletier à S\^{}t pierre\\
et à été pourvû d'une Commission\\
de maitre Connonier audit departem\^{}t\\
\& à continueé Jusqu'es à la prise de\\
L'isle quil est repassé en france avec sa\\
Compagnie \& Licencié à Rochefort\\
L'année 1762.\\
Lan 1763 en Consequence des ordres\\
du ministre il est repassé à la MSq̃ue\\
et il a été placé Garde d'artillerie\\
a S\^{}t Pierre, ou il a Continué son service\\
Jusqu'es à ce Jour.
\pend\pstart
\\
Pendant le susdit tems il à assisté à la\\
Bataille de Plaisance, à la Retraitte\\
de Tortone ou il fut prisoniés de guerre\\
et à la mnartinique il à essuyé les deuxc\\
dernieres Guerres dont à la Suitte de la\\
premier il à Suby le fort de l'isle.\\
Comme il Paroit L'autre part.\\
et depuis son retour il a continué son\\
service Jusqu'es à ce Jour ce qui le porte\\
à solliciter la medaille qui servira\\
de Titres à son fils agé de 18 ans pour\\
pouvoir meritter Dans la Suitte les bienfaits\\
du Roy comme fils d'un ancien\\
Serviteur※\\
JB Jolÿ\\
Nous commendant en chef lad\^{}t\\
de la martinique et Isle du vent\\
certifions que Le S\^{}r joly merite\\
par son zèle et ses anciens services\\
(dont les certificats nous ont été\\
présenté) La preuve quil demande\\
grandeur
\pend\pstart
\\
Le sieur Joli me parait digne, par Sa Conduite\\
et ses Services de la grace quil demande à M\^{}r le Marechal\\
à S\^{}T Pierre, le 25. 7\^{}bre 85. Du Puget※\\
vu bon\\
Maupas
\pend
\endnumbering\beginnumbering\section{FRCAOM06\_COLE\_241349C\_0115}
\vspace{0.5cm}\noindent
\textit{mg}
\footnotesize \\
Artillerie des Colonies\\
Le 11. Mars 1786.\\
Médaille d'or pour\\
Garde d'Artillerie de la\\
Martinique ※
\normalsize \pstart
\\
A M. de Cotte Directeur de la monnaye des\\
médailles\\
A Paris\\
Le Roy vient, M. d'accorder une médaille d'or\\
à un Employé de L'artillerie à la martinique;\\
Je vous prie en conséquence de m'en adresser une\\
des dernieres frappées. en observant qu'elle ait sa\\
belliere. Vous voudrés bien \^{}en même tems\\
m'indiquer le prix de cette médaille, afin que\\
je passe pourvoir à votre remboursement\\
de La Martinique\\
Au S\^{}r Joly Garde d'artillerie à S\^{}t Pierre\\
à La Martinique\\
Je vien, M, de rendre compte au Roy de le\\
L'ancienneté de vos services, ainsi que du zele et\\
de L'intelligence avec lesquels vous avés toujours\\
rempli les fonctions dans vous avés éré chargé\\
Sa Majesté a bien voulu vous accorder une\\
médaille d'or qu'elle vous permet de porter\\
comme une marque de Sa satisaction et de\\
La distinction que vous êtes acquise, et\\
j'adresse cette médaille à M. de Senneville\\
et je charge en même tems de vous en décorer ※\\
M. de Senneville ⟦⟧Lient\^{}t Colonel\\
Command\^{}t en chef L'artillerie\\
à la Martinique\\
Je vous préviens, M, que le Roy vient d'après\\
La proposition de M. de Mauson, d'accorder\\
au S\^{}r Joly Garde d'artillerie à S\^{}t Pierre\\
La médaille d'or par récompense de ses\\
services; je la lui annonce avec
\pend
\endnumbering\beginnumbering\section{FRCAOM06\_COLE\_241349C\_0116}\pstart
\\
La permission de la porter; mais vous le trouverés\\
cy jointe; et je vous prie de l'en décorer ※\\
M. Le V\^{}te de Damas Gouverneur\\
a La Martinique\\
J'ai L'honneur de vous prévenir, M, que le Roy\\
vient d'accorder au S\^{}r Joly Garde d'artillerie\\
à S\^{}t Pierre, La médaille d'or pour récompense\\
de ses ⟦⟧services. Je La lui annonce avec\\
La permission de la porter; mais je l'adresse à\\
M. de Senneville et je le ⟦prie⟧ charge d'en\\
décorerdécorer cet employé※
\pend
\endnumbering\beginnumbering\section{FRCAOM06\_COLE\_241349C\_0117}
\vspace{0.5cm}\noindent
\textit{mg}
\footnotesize \\
Artillerie\\
des Colonies.\\
Le 6 avril 1786.\\
F. env le 10 avril
\normalsize \pstart
\\
Mémoire\\
M. de Cotte Directeur des Monnoyes\\
adresse à Monseigneur la Médaille d'or qu'il\\
vient d'accorder au S\^{}r Joly Garde Magazin\\
d'artillerie au Fort S\^{}t Pierre, de la Martinique,\\
Cette Médaille monte à... 212\^{}lt\\
Monseigneur est Supplié\\
d'approuver qu'il soit donné des ordres\\
pour le remnboursement de cette somme\\
à M. de Cotte.\\
Approuvé
\pend
\endnumbering\beginnumbering\section{FRCAOM06\_COLE\_241349C\_0119}
\vspace{0.5cm}\noindent
\textit{mg}
\footnotesize \\
Dup\^{}a\\
Martinique\\
M. de Vaidre\\
30. 9\^{}bre
\normalsize \pstart
\\
Monseigneur\\
M Damerville\\
J'ai reçû la Lettre que vous m'avez fait l'honneur\\
de m'écrire en date du 11 mars, dans laquelle\\
était incluse la Médaille d'or, dont vous m'avez\\
odronner de décorer le S\^{}r Jolly, Garde d'artillerie\\
à S\^{}t Pierre\\
Je suis avec respect\\
Monseigneur\\
Vôtre très humble et très\\
obeissant serviteur.\\
Senneville
\pend\pstart
\\
Son Roy\^{}al MSque ce 15 août 1786
\pend
\endnumbering\beginnumbering\section{FRCAOM06\_COLE\_241485A\_0161}\pstart
\\
Moisset\\
(Jean)\\
Soldat à S\^{}te lucie\\
+1763\\
Colonies\\
E 314.
\pend
\endnumbering\beginnumbering\section{FRCAOM06\_COLE\_241485A\_0162}\pstart
\\
Joint a la lettre de M. de Jumilliac du 31. X\^{}bre 1764\\
Colonies\\
Hospital de S\^{}te Lucie\\
Mois\\
d'aout 1763\\
Etat des officiers et soldats Morts\\
a l'hopital de S\^{}te Lucie pendant Le mois\\
D'aout 1763\\
Scavoir\\
Moisset dit S\^{}t Julien
\pend\pstart
\\
Noms Noms\\
des\\
Regimens\\
Royal la marine
\pend\pstart
\\
Noms\\
des\\
Compagnies\\
Bergues
\pend\pstart
\\
Noms de Baptêmes\\
et de famille\\
des soldats\\
Jean Moisset d\^{}t S\^{}t Julien
\pend\pstart
\\
Nous aumonier de l'hopital de S\^{}te Lucie Certifions\\
L'etat cy dessus veritable au marinage de S\^{}te Lucie.\\
Le\\
Jour du mois de\\
1763 ※\\
Gerhoypte\\
Vu et Certifié véritable Par nous\\
Commissaire des gardes chargé de la\\
jour du d\^{}t Hop\^{}al\\
Desamur
\pend\pstart
\\
Lieux\\
de leur\\
naissance\\
S\^{}t Julien de Bousnaut\\
Ju\^{}pn de nantes en Bretagne
\pend\pstart
\\
Jour\\
de la\\
mort\\
31.
\pend\pstart
\\
Total des soldas...\\
1...
\pend\pstart
\\
Vu par nous intendent\\
de S\^{}te Lucie\\
Chandru
\pend
\endnumbering\beginnumbering\section{FRCAOM06\_COLE\_241553A\_0085}\pstart
\\
Racolet\\
(Jean)\\
soldat en Guadeloupe\\
1771\\
Colonies\\
E 344\^{}bis
\pend
\endnumbering\beginnumbering\section{FRCAOM06\_COLE\_241553A\_0086}
\vspace{0.5cm}\noindent
\textit{mg}
\footnotesize \\
Bordeaux\\
acquit de plus\\
emple décharge\\
19. 16. 8.
\normalsize 
\vspace{0.5cm}\noindent
\textit{mg}
\footnotesize \\
quatre a cette deux\\
cents quaranthuit
\normalsize \pstart
\\
Racolet.\\
Colonies 1771.\\
La Guadeloupe\\
Moitié ensus de solde accordée\\
aux Troupes de terre servant\\
aux Colonies\\
au nommé Jean Racolet dit Jolicoeur soldat au\\
Regiment de Vexin embarqué a la Guadeloupe\\
sur le navire le Scipion pour la moitié ensus de sa\\
solde des mois entiers de Janvier et fevrier 1771 et des\\
10. premiers Jours demars suivant etant débarqué en ce\\
port le 11. du dit mois à 2\^{}s. 10\^{}d par jour. 9\^{}lt 18\^{}s. 4\^{}d\\
au nommé Joseph fourmier soldat aud\^{}t\\
regiment idem pour idem ... 9. 18. 4.\\
19\^{}lt. 16\^{}s. 8\^{}d\\
Je prie Monsieur Gerbier de sa Corbinage tresorier des\\
colonies en ce port de payer aux soldats cy dessus.\\
La somme de dix neuf livrer seize sols huit deniers\\
pour les causes qui y sont enoncés laquelle somme\\
lui sera nreboursée par M. Baudard de S\^{}te James\\
trèsorier général des Colonies en lui remettant Le present\\
controllé qui sera par lui envoyé a son Commis à la\\
Guadeloupe. a Bordeaux Le 11 Mars 1771 ※\\
Vincent\\
Vu au Controlle\\
Corbin
\pend
\vspace{0.5cm}\noindent
\textit{mg}
\footnotesize \\
Bon à payer\\
Sondans\\
19. 10. 4
\normalsize 
\endnumbering\beginnumbering\section{FRCAOM06\_COLE\_241584A\_0141}\pstart
\\
Roule\\
(Jean)\\
Soldat déserteur à la Guadeloupe\\
1780\\
Colonies\\
E358
\pend
\endnumbering\beginnumbering\section{FRCAOM06\_COLE\_241584A\_0142}
\vspace{0.5cm}\noindent
\textit{mg}
\footnotesize \\
Roulle Jean\\
Jugement\\
par contumace\\
contre un Dé-\\
serteur.
\normalsize \pstart
\\
DE PAR LE ROI.\\
Vu par le Conseil de Guerre assemblé à La Basseterre\\
Guadeloupe par ordre de M\^{}r de Repentigny, brigadier\\
des armées du Roy, Colonel du Reg\^{}t de la\\
Guadeloupe, faisant les fonctions de Lieut de Roy le procès\\
extraordinairement instruit au nommé Jean Roule dit Roule\\
fils de Jean et de Guillemine Morin, natif de S\^{}te Jeté Jurisd\^{}t de\\
Lombale prov\^{}ce de bretagne taille 5 pieds 1 p\^{}ce 6.l. agé de 30 ans\\
visage rond bazané marqué de petite vérole, front large, yeux\\
roux, nez et bouche moïenne, cheveux et sourcils bruns.\\
accusé d'avoir déserté le douze aout ... 1780\\
de la Compagnie de la Marre\\
au Régiment d'Armagnac ... fugitif \& contumax;\\
l'information du dixhuit aout 1780 le récolement\\
des témoins du dit jour et an \& les conclusions\\
du sieur Jean de Cherponne C\^{}e A\^{}e Major au Regim\^{}t\\
de la Guadeloupe... ledit Conseil de Guerre a\\
déclaré la contumace bien \& duement instruite; en consé-\\
quence déclare ledit Jean Roule, dit Roule,\\
atteint \& convaincu du crime de désertion\\
\& pour réparation d'icelui l'a\\
condamné et condamne à la Chaine pour huit années\\
consecutives\\
Fait à Labasseterre... le Dixhuit\\
du mois d'aout ... mil sept cen quatre-vingt\\
Le bateau\\
Le ch\^{}e trapeur\\
Lepour\\
Le Ch\^{}er de parfour\\
Batalle\\
Repatigny\\
Villé\\
Le preésent Jugement a étà lu à la Garde montante, le 20 aout 1780\\
Joubeur
\pend
\endnumbering\beginnumbering\section{FRCAOM06\_COLE\_241601A\_0886}\pstart
\\
Personnel\\
de Salvigny.\\
(Frédérick)\\
Soldat en Guadeloupe\\
1780\\
Colonies\\
E⟦365⟧\\
364
\pend
\endnumbering\beginnumbering\section{FRCAOM06\_COLE\_241601A\_0887}
\vspace{0.5cm}\noindent
\textit{mg}
\footnotesize \\
N°10\\
Inspection de 1780.\\
depence Extraordinaire\\
de 120\^{}lt argent de france
\normalsize \pstart
\\
Régiment D'armagnac\\
Etat des avances faites par la caisse du dit Regiment pour les Engagements et\\
Embausages des Recrues fait par le Soin de L'etat major depuis le p\^{}r juin 1779 jusqu'au\\
1\^{}er juin 1780. dont il demande le Remboursemens net sans aucune retenüe\\
Savoir
\pend\pstart
\\
Nom de\\
Compagnie\\
Datte de son\\
Somme payée et\\
Bapteme et familles Guerre\\
avancée par la\\
Engagement\\
âge\\
caisse du Régiment\\
Taille\\
D'Jusse\\
Frederich de Salvigny\\
Salvigny\\
le 8\^{}e 9\^{}bre 1779\\
120\\
20\\
5p 1p 6L\\
Total argent de france 120\^{}lt\\
fait et arêtté au Conseil d'administration Basseterre Le\\
Luiravoz monval Fuyseux\\
Rostaing Lagarde\\
Nous chef d'escadre Gouverneur, Lieutenans General pour sa Majesté des Isles Guadeloupe\\
et dependances, Vu le present Etat montans à la Somme de Cent vingt Livres argent de france\\
Détaillé cy dessus pour L'engagement et embausage des Recrües par le soin de L'Etat\\
Major dud\^{}t Régiment et avances par la Caisse ordonnons que le Remboursement lui en\\
soit fait net sans aucune Retenües\\
Basseterre Isle Guadeloupe Le 1\^{}er Juin 1780\\
Helte darbaud
\pend
\endnumbering\beginnumbering\section{FRCAOM06\_COLE\_241619A\_0803}\pstart
\\
Suzanne\\
(Noël)\\
Soldat à St-Domingue\\
1751\\
Colonies\\
E 374.
\pend
\endnumbering\beginnumbering\section{FRCAOM06\_COLE\_241619A\_0804}
\vspace{0.5cm}\noindent
\textit{mg}
\footnotesize \\
1751. 18 Mars\\
Suzanne (Noël)
\normalsize \pstart
\\
Jugement d'un Conseil de Guerre tenu à S\^{}t Domingue\\
qui condamne le n\^{}é Suzanne à être pendu pour avoir\\
débauché des Soldats de la Compagnie de Guichon.\\
Lu par le Conseil de Guerre assemblé Chéz Monsieur\\
Le Comte de Conflans Chevalier de L'ordre Royal et\\
militaire de S\^{}t Louis Gouverneur Et Lieutenant\\
Général pour Le Roy des Isles Francoises de L'amérique\\
sous Le Vent, Compozé de mon dit Sieur qui y à rezidé,\\
de messire Jean Baptiste Laporte De Lalanne, Conseiller\\
du Roy en ses Conseils, Commissaire Général de\\
police, finance, et de la marinne, faisant fonction\\
D'Intendant au dit paÿs, de messire henry\\
de massip Lieutenant de Roy Commandant à\\
Leogane De La tour chevalier de L'ordre Royal et\\
Militaire de S\^{}t Louis, Capitaine d'Infenterie olivier\\
de vieux chatet Capitaine D'Infanterie de Mitton\\
de Seneville Capitaine D'Infanterie de Beaupré Cap\^{}ne\\
D'Infanterie, De Sellierre Capitaine D'Infanterie\\
De Massip Lieutenant D'Infanterie; La Plainte faitte à\\
Monsieur de Conflans par Monsieur de Guichen\\
demendeur et accusateur Contre Le nommé Noël\\
Suzanne accusé D'avoit séduit et debauché plusieurs\\
Soldats de sa Compagnie au Bas de laquelle est\\
L'ordre pour en être Informé, en conséquence dequoy\\
Information, recolement, et Confrontation, et Interrog\^{}re\\
de l'accusé sur La Sellette, les Conclusions du Sieur\\
de Gairose Major données, tout Considéré et\\
murement examiné, Le Conseil de Guerre à Trouvé\\
Le dit nöel suzanne atteint et Convaincu d'avoir\\
séduit et debauché plusieurs soldats de sa
\pend
\endnumbering\beginnumbering\section{FRCAOM06\_COLE\_241619A\_0805}\pstart
\\
Compagnie pour reparation de quoy et attendû\\
qu'il n'y à ny prison ny Garnison au Port au Prince\\
suffisante, Lieu ou le délit a été Commis L'a Condamné\\
à être Conduit à La tête des Troupes ce Cette\\
Ville qui seront Rangée en Bataille et Livré à\\
L'executeur la haute Justice pour être pendû\\
et etranglé jusqu'à ceque mort en s'en suive\\
et attendûs les faits Resultans en La procedure\\
Contre Le nommé Belair Tambours de la Comp\^{}e\\
De Guichen L'a condamné à assister à L'exécution\\
et à garder trois mois Le Cachot à Compter du\\
jour du present; fait en Conseil à Leogane Le\\
dix huit mars mil sept cent cinquante un\\
signéz à la minutte de Conflans, La porte la lanne\\
De massip de La Tour, olivier de vieux Chatel,\\
mithon de Seneville, Beaupré, Serierre, Le Ch\^{}er de massip\\
Gairolle, et Cheverry Greffier ※
\pend
\endnumbering\beginnumbering\section{FRCAOM06\_COLE\_241652A\_0132}\pstart
\\
Voidant\\
(Pierre-Michel)\\
Déserteur des colonies\\
1758\\
Colonies\\
E 389\^{}bis
\pend
\endnumbering\beginnumbering\section{FRCAOM06\_COLE\_241652A\_0133}
\vspace{0.5cm}\noindent
\textit{mg}
\footnotesize \\
May 1758.\\
Voydant\\
michel\\
a M. Gignoux
\normalsize \pstart
\\
Le s\^{}r Darmancourt Lieutenant de\\
Maréchaussée a Nantes\\
Envoyé les interrogatoires de six\\
particuliers qui ont été arretés par la\\
Brigade de Chevreux, et qui sont déserteurs\\
d'une recrüe de 25. hommes destinée\\
pour les colonies francoises\\
Vu de ces particuliers nommé Pierre\\
michel Voydant paroist d'autant\\
plus suspect qu'il semble avoir\\
voulu cacher le lieu de son origine\\
s'étant dit d'abord de namur, et\\
ensuite de Geneve, et n'ayant pas voulu\\
rendre raison de la conduite qu'il a\\
tenüe a Paris, ny des endroits où\\
il y avoit demeuré.\\
Deux autres de ces prisonniers\\
nommés Sebastien Géraud, et Noël
\pend\pstart
\\
mald\^{}eLe\\
Mémoire
\pend
\endnumbering\beginnumbering\section{FRCAOM06\_COLE\_241652A\_0134}\pstart
\\
aisé ont dit avoir servi dans les\\
Régimens de haynaut, et de Belsunce,\\
d'ou ils ont déserté et sont passé au\\
service d'Espagne qu'ils ont aussy\\
abandonné pour revenir en france a\\
la faveur de L'amnistie.\\
Il ne paroist pas qu'il y ait rien de\\
capital sur le compte des trois derniers\\
qui sont des raudeurs et des vagabonds\\
de profession.\\
a L'Egard du Commandant de la\\
Brigade de Chevreuse il paroist qu'il\\
a manqué a ce qu'il devoit aud\^{}t\\
Darmancourt en négligeant de\\
l'informer de la détention de ces\\
prisonniers, et pour ne les avoir pas\\
fait transférer a Nantes, comme\\
il  l'avoit dû※
\pend
\endnumbering\beginnumbering\section{FRCAOM06\_COLE\_241652A\_0135}
\vspace{0.5cm}\noindent
\textit{mg}
\footnotesize \\
Pour Copie\\
Voidant\\
pierre michel
\normalsize \pstart
\\
Extrait de L'interrogatoire d'un\\
particulier très suspect; soupconné\\
de desertion, arresté par la brigade\\
de marechaussée de Chenrense Les\\
22. de ce mois, Suby devant nous\\
Charles Benoist D'armencourt\\
Lieutenant de la marechaussée\\
a la residence de nantes,\\
à cet effet transporté audit Cheureuse\\
accompagné de notre Greffier
\pend
\vspace{0.5cm}\noindent
\textit{mg}
\footnotesize \\
Homme très suspect.\\
N\^{}a paroist avoir\\
36 ans.
\normalsize \pstart
\\
ordinaire, ainsy qu'il ensuit.\\
Du 28. avril 1758.\\
Premierement Interrogé de son nom, Surnom, age,\\
qualité, demeure le lieu de sa naissance.\\
A dit s'apelle Pierre michel Voidant, agé de\\
23. ans, charpentier de sa proffession, sans aucune\\
demeure fine, natif de namur en flandres dit\\
presentement estre natif de Geneve paroisse S\^{}t Gervais.\\
Interrogé s'il à pere et mere et sil est marié.\\
a dit quil à pere et mere, qui demeurent audit\\
Geneve, le Cabaretier, qu'il n'est point marié.\\
Interrogé ou il alloit Lorsqu'il a été arreté.\\
A dit qu'il alloits a Versailles poir y chercher der\\
Louvrage.\\
Interrogé de quel endroit Il venait avant sa capture.\\
A dit qu'il venoit D'etampe, ayant quitté une\\
recrüe dont il étoit, s'étant engagé a Paris Le 14.\\
de ce mois, au S\^{}r Labarthe sergent des Colonnies\\
françoises Logé rüe de la Vannerie près la Greve.\\
Interrogé Pourquoy il a quitté La troupe.\\
a Dit que le d\^{}t sergent luy avoit promis\\
12\^{}lt mais que ne luy ayant pas donné, cela la\\
déterminé aquitter la recrüe avec cinq de ses\\
Camarades
\pend
\endnumbering\beginnumbering\section{FRCAOM06\_COLE\_241652A\_0136}\pstart
\\
Interrogé sil à emporté L'uniforme de la troupe.\\
A dit quil n'a emporté qu'un sacq, une paire\\
de soulliers, et une paire de Guestres, qui luy\\
avoit été donné avant que de partir.\\
Interrogé Combien Il y a de tems quil est\\
sorty de Geneve pour entrer en france.\\
A dit quil y a environ en france.\\
Interrogé quelle Conduitte Il y a tenû dans\\
les endroits ou Il a esté et chez qui Il a demeuré.\\
A dit quil est venû à Paris Il y a un an,\\
ne peut nous dire la Conduitte quil à tenüe, ny\\
Les endroits ou Il a demeuré a Paris,\\
Interrogé sil a esté repris de justice\\
a dit que non.\\
Interrogé sil a dir la Vérité\\
a dit que oùÿ.\\
Lecture entiere a luy faite du present\\
Interrogatoire, a dit que ses reponses Contiennent\\
Verité, quil y persiste et a declaré ne scavoir\\
Ecrire, signé Darmancourt et Le Roy.\\
Homme trés suspect, ne voulant nous\\
dire rien, sur toutes les questions que nous Luy\\
avons demandé.
\pend
\endnumbering\beginnumbering\section{FRCAOM06\_COLE\_241652A\_0137}
\vspace{0.5cm}\noindent
\textit{mg}
\footnotesize \\
au moins 36 ans.
\normalsize \pstart
\\
Signalement\\
Pierre Michel Voidant agé de 23 ans\\
natif de Geneve, ou de namur, Taille de\\
5. pieds, 5 pouces 6 lignes, visage plein et rond\\
beaucoup marqué de petite verolle, les yeux gris,\\
barbe, sourcils, le cheveux noirs, le front Grand,\\
point de Cheveux au toupet, bien fait dans\\
sa taille ※
\pend
\endnumbering\beginnumbering\section{FRCAOM06\_COLE\_241652A\_0427}\pstart
\\
Volant\\
(François-Henri)\\
Soldat à S\^{}t Domingue\\
1748-91\\
Colonies\\
E 389 bis
\pend
\endnumbering\beginnumbering\section{FRCAOM06\_COLE\_241652A\_0428}
\vspace{0.5cm}\noindent
\textit{mg}
\footnotesize \\
Du 15 avril 1791.\\
Ru le 26 mai\\
N°⟦38⟧\\
125
\normalsize \pstart
\\
Volant (françois Henry)\\
Colonies\\
La Société des amis de La constitution d'Orleans\\
adresse un memoire par lequel le S. Volant ancien sergent\\
du Régiment Provincial de Paris employé depuis 1763\\
dans la partie du recrutement, demande des Lettres de\\
sous Lieutenant à la Suite des troupes des Colonies.\\
MeM Les Deputés Dorleau à L'assemblée Nationale\\
attestent la Bonne conduite et l'honneteté de ce\\
recruteur.\\
Le S. Volant a d'aucun services comme\\
soldat. La demande qu'il fait d'un Brevet de\\
Sous Lieutenant à La Suite du Bataillons auxiliaire\\
le mettroit dans le cas d'avoit le Lendemain qu'il\\
lui seroit expédié La decoration militaire.\\
Ausurplus les principes établis s'opposent\\
à ce qu'il soit accordé aucun Brevet à La suite\\
et qui Ne permet pas d'accueillir la demande\\
du S. Volant. On a verifié qu'il jouit de\\
deux demi solde, L'une de 90\^{}lt sur les invalides\\
de la Marine qui lui a été accordée en 1753\\
et L'autre de 168\^{}lt sur les fonds de la Guerre,\\
pour service rendus dans le dernier département ※\\
On ne peut donner de Brevet\\
à la suite※\\
Vérifions s'il n'est pas nécessaire\\
que les 2 services soient réunies\\
en un seul Brevet※
\pend
\endnumbering\beginnumbering\section{FRCAOM06\_COLE\_241652A\_0429}\pstart
\\
M. Volant\\
Du 26 Mai 1791\\
J'ai ⟦recu⟧ M. le memoire par lequel\\
examiné\\
PJ\\
Vous ⟦demandés⟧ des lettres de sous Lieutenant\\
sollicités\\
à la suite des Troupes des Colonies\\
pour Vous mettre dans le cas d'obtenir\\
La décoration militaire. Les principes\\
établis sopposent  à ce qu'il soit\\
accordé aucun Brevet à la suite ce qui\\
ne me permet pas de mettre votre\\
demande sous Les yeux du Roi.\\
Le MX
\pend
\endnumbering\beginnumbering\section{FRCAOM06\_COLE\_241652A\_0430}\pstart
\\
M. de Vaivre\\
30 avril.\\
A Monsieur Le Chevalier de Fleurieu\\
Ministre de la Marine\\
M Sand\\
de présenter un Mémoire à Monsieur Le Chevalier\\
Le S\^{}r François henry Volant a eu l'honneur\\
de Florieu, le 17 mars 1791, dans lequel est détaillé\\
43 années de services prouvés par des Certificats,\\
par lequel Mémoire, le supliant à demandé en\\
considération de ses services qu'il lui soit accordé\\
le Rang de Sous Lieutenant dans la Troupe du\\
recrutement des Colonies ou il est employé dans\\
cette même qualité à Orléans; ce qui le mettra a\\
même d'obtenir la Décoration Militaire qu'il se\\
croit susceptible de mériter, c'est la seule\\
Récompense que le Supliant demande et il en\\
conservera un précieux souvenir.
\pend
\endnumbering\beginnumbering\section{FRCAOM06\_COLE\_241652A\_0431}\pstart
\\
Orléans le 2 Mars 1791,\\
l'an 2\^{}e de la liberté
\pend
\vspace{0.5cm}\noindent
\textit{mg}
\footnotesize \\
M. de Vaivre,\\
25 mars\\
PJ
\normalsize \pstart
\\
Monsieur\\
M Sand\\
La Société des amis de la Constitution séante à orléans\\
compte parmi ses Membres, M. Voland chargé depuis\\
longtemps de recruter en cette ville. l'admission de M\\
voland dans la société vous annonce qu'il a été reconnu\\
pour un citoyen Estimable. nous savons qu'il sollicite\\
son avancement, et la recompense de ses services. la société\\
apprendra avec plaisir que vous aurez accordé à M voland\\
la Justice qu'il a droit d'attendre.\\
Nous sommes avec le plus sincere attachement\\
Monsieur\\
Les Membres de la société\\
des amis de la constitution\\
Séante à orléans\\
Moisart President\\
Ladureau secrétaire\\
Prouvinete M secretaire
\pend
\vspace{0.5cm}\noindent
\textit{mg}
\footnotesize \\
M le Ministre de la Marine
\normalsize 
\endnumbering\beginnumbering\section{FRCAOM06\_COLE\_241652A\_0432}\pstart
\\
A Monsieur\\
Le Ch\^{}er De Fleurieu\\
Minsitre de la Marine\\
Memoire\\
François henry Volant entré au service dans\\
les Troupes de S\^{}t Domingue en qualité de soldat en 1748\\
y a reçu des blessures graves qui avoient été jugées incurables,\\
suivant les Certificats des Chirurgiens dont il a justifié à\\
son retour en france, dans le courant de l'année 1753.\\
A cette époque, et en considération desd. blessures\\
sa Majesté lui a accordé sa demie solde sur les fonds\\
des invalides de la Marine à raison de 7\^{}lt. 10\^{}s. par\\
mois. Le Brevet de ce traitement est du 23 fevrier 1754\\
et le payement a commencé du premier Juillet 1753 la\\
Copie dud. Brevet est ci jointe.\\
La Jeunesse du S. Volant, de grands soins et les\\
Eaux de Barége lui procurèrent assez promptement une\\
guérison aussi radicale qu'inesperée; dès le 5. avril\\
de la même année 1754. Son inclination naturelle le fit\\
rentrer au service dans le Régiment Suisse d'halletville,\\
où il est resté quatre ans.\\
A la reforme de ce Corps, il entra de suite dans\\
celui d'Infanterue française de Buissac avec lequel
\pend
\endnumbering\beginnumbering\section{FRCAOM06\_COLE\_241652A\_0433}\pstart
\\
il a fait les premieres Campagnes d'hanorve.\\
Blessé une Seconde fois d'un Coup de feu à la\\
bataille de Rosbach il n'en continua pas moins de\\
servir jusqu'en 1759 qu'il obtient son congé absolu.\\
Le 16 avril 1760 il est rentré dans le Regiment\\
Provincial de Paris où il a fait le reste des Campagnes\\
d'hanorve en qualité de Sergent.\\
Volant s'est employé sur le champ au recrutement\\
Lors du licenciement de ce Corps à la paix le S\^{}r\\
des Troupes du Roi, tant pour le Département de la\\
Guerre, que pour celui de la Marine sous les ordes\\
de M. Agobert. Le Certificat cijoint de ce Lieutenant\\
Colonel et celui de la Municipalité d'Orléans où ledit\\
S\^{}r Volant reside depuis 18 ans ne laissent rien à désirer\\
sur le zèle le désinteressement et le Succès avec\\
lesquels il y fait ses opérations de Reserves en qualité\\
d'officier préposé.\\
En 1778 ayant justifié par de bons certificats qu'il\\
servoit depuis 30 ans revolu, dont plus de la moitié en\\
qualité de sergent, il obtint des bontés du Roi, en cette\\
dernière qualité la récompense Militaire à raison\\
de 168\^{}le par an: Le Brevet est du 30 x\^{}bre 1778 et\\
la Copie en est ci jointe.\\
Il resulte de l'Exposé cidessus et des pieces y\\
jointes que le S\^{}r François henry Volant sert avec\\
zèle et distinction depuis 43 ans.\\
Qu'il a été très dangereusement blessé au Service\\
dans sa premiere Jeunesse et qu'il y est néanmoins
\pend
\endnumbering\beginnumbering\section{FRCAOM06\_COLE\_241652A\_0434}\pstart
\\
rentré aussitôt que son rétablissement le lui a permis.\\
Enfin, que depuis 18 ans il a eu la qualité\\
dôfficier préposé donnée aux principaux Recruteurs\\
par les ordonnances du Roi concernant les Régimens\\
de Recrües.\\
Mais cette qualité qui Sembloit le placer parmis\\
les autres officiers dits de fortune et qui lui a en effet\\
donné le rang dans la Société ne la point rendu\\
susceptible de la Décoration militaire que viennent\\
d'obtenir plusieurs de ses Camarades dont la plüpart\\
ont beaucoup moins servi que Lui.\\
Il supplie en conséquence Monsieur Le Chevalier\\
de Fleurieu de vouloir bien lui accorder les Lettres\\
de Sous Lieutenant à la suite des Troupes des\\
Colonies au recrutements desquelles il continüe de se\\
rendre utile avec le même zèle et toujours sous les\\
ordres de M. Agobert.\\
Il en sera pénetré de la plus respectueuse\\
reconnaissance ※\\
Les dépots dorleans a L'assemblé nationale attestent\\
a Monsieur de fleurieu La Bonne conduite et l'honneteté\\
toujours soutenue De M\^{}r volant et prient Monsieur\\
Le Ministre de la marine de vouloir Bien accueillir\\
la demande Seurrat L'haboulay\\
Lefort\\
⟦⟧ Recouron
\pend
\endnumbering\beginnumbering\section{FRCAOM06\_COLE\_241652A\_0435}\pstart
\\
Copie du Brevet de la demie solde\\
du S\^{}r Volant du 23 fevrier 1754\\
De Par le Roy\\
Sa Majesté ayant accordé la demi Solde\\
à françois henry Volant soldat dans les\\
Troupes de S\^{}t Domingue dont les infirmités le\\
mettent hors d'Etat de continuer ses Services suivant\\
les certificats qu'il a produits; L'intention de sa\\
Majesté est qu'il soit employé dans les Etats de\\
demi solde des invalides de la Marine autrement à\\
Paris à raison de Sept Livres dix Sols par mois,\\
à commencer du premier Juillet 1753. fait à\\
Versailles le 23 fevrier 1754. signé Louis et\\
plus bas Bouille ※\\
Pour Copie conforme à l'Original\\
Voland
\pend
\endnumbering
\end{pages}
\end{document}
